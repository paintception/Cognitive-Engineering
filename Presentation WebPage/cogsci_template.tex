\documentclass[10p,letterpaper]{article}

\usepackage{cogsci}
\usepackage{pslatex}
\usepackage{apacite}


\title{The Importance of Interaction in Information Visualization Systems}
 
\author{{\large \bf Matthia Sabatelli (m.sabatelli@student.rug.nl)} \\
  Rijksuniversiteit Groningen\\Department of Artificial Intelligence\\ Nijenborgh 4,
  9747 AG Groningen\\ The Netherlands}

\begin{document}

\maketitle


\begin{abstract}

Cognitive Engineering is the name of a discipline that combines two terms that are very different between each other. On the one side, the first one relates to the world of the human mind and all the processes that guide it, while on the other one, the second term is strongly connected to the development of mechanical and technical techniques that support the creation of technological applications. Although this juxtaposition may seem counter intuitive in this paper I show how a Cognitive Engineering approach is needed when building appropriate Information Visualization Techniques and how it is linked to the importance of developing tools that support interactive processes.       

\textbf{Keywords:} 
Cognitive Engineering, Information Visualization, Interaction, Interactive Techniques, Big Data
\end{abstract}


\section{Introduction}

During the October of 2012 Dr. John Barrett, head of the Embedded Systems Research Center of Cork has given a TED talk in which he estimates that there were almost 4000 exabytes 
of information saved on the cloud. In order to understand the meaning of how large this quantitative of information is, it's possible to correspond this measure to eighty stacks of books that go from the Earth to Pluto. If until fifty years ago academics of all different disciplines were struggling in order to gain some data in order to do their researches on, nowadays this aspect doesn't seem to be an issue any more. Computers and smartphones are only two out of the multiple technological systems that are able to collect information 24/7, information that is safely saved on cloud servers and that due to its incredibly large size has lead to the juxtaposition of the terms \textit{Big \& Data}.\\
If at the one side the problem of collecting information has been solved thanks to the technological developments of the last decades, the formation of Big Data has lead to a likewise dilemma that researchers have to deal with: Information Visualization. Information Visualization, or \textit{Infovis} *Ref*, is a very complex research domain that has as main goal the representation of large-sized information that is mostly saved on the cloud. A lot of research that tries to solve this problem from a pure engineering perspective has been done, computer scientists and engineers have built multiple complex tools and representation strategies that are able to display large amounts of information. However, developing these kind of tools by considering only technical aspects, isn't enough to make the information that is displayed usable and easily understandable to the person who is watching at it. In fact, in order to build a successful information visualization system there is need to understand the cognitive mechanisms that guide the user when it's interacting with the tool, and only in a second moment, by taking these cognitive knowledge into consideration, implementing correct visualization strategies.\\
The main objective of this work is to prove how important it is, to understand the cognitive mechanisms behind the UX of a person that is trying to infer some knowledge from data that is displayed in front of him. In order to do this the paper will be divided into 2 parts. In the first one an analysis of the overall visualization process will be presented by focusing on how this method relates to some important cognitive science aspects. Once this is done multiple examples of different visualization techniques will be explained. The goal of this part is double, on the one side it aims to present some of the most recent technological achievements that have made it possible to represent large amounts of information, while on the other side, it aims to prove, how large part of the techniques that are considered as successful take into consideration the process of \textit{Interaction}, an aspect that is the link between the digital stored data and the human mind, and that literature doesn't consider as relevant as the visualization part.


\section{Insight \& Mental Models}

In this section two important aspects related to the field of cognitive science are presented that both have an impact in the development of Information Visualization System. They are introduced by in *chapter*.\\  
The first concept is defined as \textit{Insight} and is strongly related to the human vision system. Between the five senses, vision is the one that is the most capable of rapid parallel processing and pattern recognition, thanks to the millions of photoreceptors it's in fact possible to acquire large amount of visual information very quickly. However, this isn't enough to understand the data that is presented, the human mind has in fact to deal with a higher level process that leads to the extraction of knowledge from the information that is visualized. If this process turns out to be successful, which means that the user gains some comprehension from the data, it's possible to assert that it has acquired insight. When building a Visualization System this is crucial, in fact, providing the user with insight corresponds to the final goal that \textit{Infovis Systems} have to achieve.\\
The importance of this concept has also been underlined in work *DefiningInsitforVisualAnalytics* where the authors argue that the purpose of visualization is indeed insight, and that appropriate visualization tools should enable the user to discover this concept. However, the authors also highlight the fact that the scientific community has been slow to build on this idea and a commonly accepted definition doesn't exist yet. In order to solve this issue, the paper presents two different definitions of the concept that both have to be taken into account in visual analytics. The first one takes inspiration from the standard cognitive science explanation of the concept and defines insight as the moment of enlightenment that leads to a state in which it's possible to solve a problem that previously was unsolvable. The second one sees insight as a simple advance of knowledge and gain of information. Both definitions turn to fit perfectly to a scenario of a system that is displaying some data to a user that has to understand it, in fact an appropriate visualization tool shouldn't take too long to lead the user to the moment of enlightenment and should do this by providing him with a new piece of information.\\
In order to subsidize the user to get insight a second concept presented by the **chapter** turns to be relevant, it regards \textit{Mental Models}. According to *MentalModelsConceptsRevisited* a Mental Model is a personal explanation of someone's thought process about how something works in reality. In fact, it can be defined as an internal representation of the surrounding world that helps humans to understand a particular phenomenon. According to *chapter* Mental Models turn to be very important in information visualization since the data that is displayed should be in line with the internal mental models of the users, only if this is the case the users will get insight about the data.\\
The importance of understanding the Mental Models of the users in order to facilitate the use of visualizations has also been pointed out in work *MentalModels,VisualReasoningndInteraction*. In this paper the authors assert that most of the current research in \textit{InfoVis} is focused on external visualization and almost no research at all has been done on internal representations. The development of appropriate visualization tools has to deal with the understanding of the internal visualization skills of the users that are as much important as the creation of external visualization tools.\\
 	
\section{The Design Process \& The Visualization Pipeline}

\section{Examples of Interactive Techniques}

\section{Future Directions \& Conclusion}
possibile collegamento con machine learning

\section{References}

\nocite{ChalnickBillman1988a}


\bibliographystyle{apacite}

\setlength{\bibleftmargin}{.125in}
\setlength{\bibindent}{-\bibleftmargin}

\bibliography{CogSci_Template}


\end{document}
